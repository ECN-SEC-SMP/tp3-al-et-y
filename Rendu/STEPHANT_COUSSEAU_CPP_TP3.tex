% Hyperliens pour doc
% https://www.overleaf.com/learn/latex/Positioning_images_and_tables <== Images, tableaux
% https://www.physicsread.com/latex-mu-and-nu-symbol/ <== Formules maths
% https://www.overleaf.com/learn/latex/Code_listing <== Morceaux de code

\documentclass[11pt]{article}

% Librairies
\usepackage{geometry}
\usepackage{graphicx}
\usepackage{wrapfig}
\usepackage{subcaption}
\usepackage{fancyhdr}
\usepackage{listings}
\usepackage{xcolor}
%\usepackage[french]{babel}

% Setup du document (paramètres)
\geometry{hmargin=2.3cm,vmargin=3cm}

\definecolor{codegreen}{rgb}{0,0.6,0}
\definecolor{codegray}{rgb}{0.5,0.5,0.5}
\definecolor{codepurple}{rgb}{0.58,0,0.82}
\definecolor{backcolour}{rgb}{0.95,0.95,0.92}

\lstdefinestyle{mystyle}{
    backgroundcolor=\color{backcolour},   
    commentstyle=\color{codegreen},
    keywordstyle=\color{magenta},
    numberstyle=\tiny\color{codegray},
    stringstyle=\color{codepurple},
    basicstyle=\ttfamily\footnotesize,
    breakatwhitespace=false,         
    breaklines=true,                 
    captionpos=b,                    
    keepspaces=true,                 
    numbers=left,                    
    numbersep=5pt,                  
    showspaces=false,                
    showstringspaces=false,
    showtabs=false,                  
    tabsize=2
}

\lstset{style=mystyle}

% Setup du document (texte)
\setlength{\parskip}{5pt}
\renewcommand{\contentsname}{Table des matières}
\renewcommand{\headrulewidth}{0.5pt}
\renewcommand{\footrulewidth}{0.5pt}


% Début de la rédaction
\begin{document}

\pagestyle{fancy}

\begin{titlepage}
\centering
{\Huge\bfseries Compte rendu CPP n°3}

\vspace{1cm}

{\huge Templates}

\vspace{2cm}

{\large STEPHANT André-Louis, COUSSEAU Yanis}

\vspace{2cm}

% Insérer les images ici
\begin{figure}[h]
    \centering
    \includegraphics[width=1\textwidth]{Ressources/LogoCN_RVB.jpg}
\end{figure}

\vfill

{Ecole Centrale de Nantes - Systèmes embarqués Communicants (SEC)}
\end{titlepage}

% Clear all headers and footers (see also \fancyhf{})
\fancyhead{}

% Set the Centre header location but do not specify O or E
\fancyhead[L]{STEPHANT-COUSSEAU}
\fancyhead[C]{SEC2}
\fancyhead[R]{17/11/2025}

% Set the Left footer location but do not specify O or E


\sloppy

\begin{center}
\rule{8cm}{1pt}
\end{center}

% Table des matières
\tableofcontents

\begin{center}
\rule{8cm}{1pt}
\end{center}

\section{Introduction}
Ce troisième TP a pour objectif de nous faire travailler une nouvelle notion en C++ : les templates.
Pour rappel, un template est une manière de simplifier un code, permettant à une fonction d'accepter différents types de paramètres en entrée.

Dans ce TP, nous allons utiliser cette notion de templates avec des classes, ainsi que des surcharges d'opérateurs. 
L'objectif est de mêler plusieurs concepts différents afin de mener à bien un projet.

Dans un premier temps, nous allons créer une classe Point template. Puis, nous coderons une classe forme, dérivée de la classe précédente.
Ensuite, nous utiliserons une classe rectangle, héritée de la classe forme.
Enfin, nous spécialiserons nos classes, puis créerons une liste de forme.

Nous conclurons ce rendu en mettant en lumière les différentes compétences sollicitées pour ce TP.

\clearpage

\section{Création d'une classe point template}
Dans cette première partie, nous allons initialiser le premier template de notre projet : la classe Point.
Pour rappel, voici le cahier des charges associé à cette classe :

\rule{4cm}{1pt}
\begin{figure}[h]
    \centering
    \includegraphics[width=1\textwidth]{Ressources/CDC partie 1.png}
    \caption{Cahier des charges partie 1}
\end{figure}

Par conséquent, on crée un fichier .hpp afin de répondre au cahier des charges :

\begin{lstlisting}[language=C++, caption=Fichier PointT.hpp]
Ctrl C + Ctrl V le code ICI
\end{lstlisting}

On ajoute ensuite des jeux de test afin de valider le bon fonctionnement du programme :

\begin{center}
\rule{16cm}{1pt}
\end{center}

\textbf{Donnée d'entrée : } NA

\rule{4cm}{0,2pt}

\textbf{Résultat attendu : } NA

\rule{4cm}{0,2pt}

\textbf{Résultat obtenu : } NA

\begin{center}
\rule{16cm}{1pt}
\end{center}

On peut donc valider que la classe fonctionne comme attendue.

\clearpage

\section{Formes géométriques abstraites}
Dans cette seconde partie, nous allons créer une classe afin de manipuler une forme, constituée de points. 
L'objectif est de réaliser une forme centrée sur un point particulier (initialisé dans la partie précédente).
Pour rappel, voici les instructions données dans le sujet de TP associées à cette classe :
\rule{4cm}{1pt}
\begin{figure}[h]
    \centering
    \includegraphics[width=1\textwidth]{Ressources/CDC partie 2.png}
    \caption{Cahier des charges partie 1}
\end{figure}

Par conséquent, on crée un fichier .hpp afin de répondre au cahier des charges :

\begin{lstlisting}[language=C++, caption=Fichier Forme.hpp]
Ctrl C + Ctrl V le code ICI
\end{lstlisting}

On ajoute ensuite des jeux de test afin de valider le bon fonctionnement du programme :

\begin{center}
\rule{16cm}{1pt}
\end{center}

\textbf{Donnée d'entrée : } NA

\rule{4cm}{0,2pt}

\textbf{Résultat attendu : } NA

\rule{4cm}{0,2pt}

\textbf{Résultat obtenu : } NA

\begin{center}
\rule{16cm}{1pt}
\end{center}

On peut donc valider que la classe fonctionne comme attendue.

\clearpage

\section{Formes géométriques concrètes}
On veut ici réaliser des formes géométriques concrètes. C'est-à-dire une forme constituée de points, avec une longueur et une hauteur donnée.
On va créer ici deux classes (rectangle et carré, avec notion d'héritage pour cette dernière).

On crée donc deux fichiers .hpp :

\begin{lstlisting}[language=C++, caption=Fichier Rectangle.hpp]
Ctrl C + Ctrl V le code ICI
\end{lstlisting}

\begin{lstlisting}[language=C++, caption=Fichier Carre.hpp]
Ctrl C + Ctrl V le code ICI
\end{lstlisting}

On ajoute ensuite des jeux de test afin de valider le bon fonctionnement du programme :

\begin{center}
\rule{16cm}{1pt}
\end{center}

\textbf{Donnée d'entrée : } NA

\rule{4cm}{0,2pt}

\textbf{Résultat attendu : } NA

\rule{4cm}{0,2pt}

\textbf{Résultat obtenu : } NA

\begin{center}
\rule{16cm}{1pt}
\end{center}

On peut donc valider que la classe fonctionne comme attendue.

\clearpage

\section{Conclusion}
Ce TP nous a permis de découvrir et d'approfondir la notion de template en C++.
Nous avons pu mettre en pratique cette notion à travers la création de plusieurs classes, ainsi que l'utilisation de l'héritage et des surcharges d'opérateurs.

Nous avons également renforcé notre compréhension des concepts liés aux classes, ainsi que du langage C++ d'une manière générale.
Ces notions nous seront importantes pour l'avenir, notamment dans le cadre du projet de fin de module à venir.
\end{document}